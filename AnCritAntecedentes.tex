\subsection{Análisis crítico de los antecedentes}

En la investigación conducida por \textcite{corona-lopezSynthesisCharacterizationPhotophysical2017} se detalla la preparación de \gls{BOSCHIBA} y su potencial aplicación como agente de tinción citoplasmtica \invitro{}; en esta investigación se sintetizó el producto por una reacción de condensación entre \iupac{2-hidroxi-1-etil naftaldehido} con derivados de anilina. Se encontró en este proyecto que al incorporar sustituyentes voluminosos, mejoraba la estabilidad de las \gls{BOSCHIBA}.

\textcite{ibarra-rodriguezOrganoboronSchiffBases2019} investigaron la aplicación de \gls{BOSCHIBA} en fluorescencia y bioimagen celular. Por su respuesta a la viscosidad, resultante en un aumento en el rendimiento cuántico, se propuso su uso para detección de células cancerígenas. Los autores mencionan que es importante para diseñar marcadores citoplasmáticos que sean fluorescentes, con propiedades fotofísicas adecuadas, fotoestables, con baja citotoxicidad y solubles en solventes polares para ser usados en biología celular moderna \invitro{} y \invivo{}; en esta investigación se lograron teñir las células cancerígenas B16F10, sin embargo, el teñido del citoplasma fue bajo, atribuido principalmente a la baja solubilidad de los compuestos; por lo que en esta investigación se propone la síntesis de \gls{BOSCHIBA} derivada de aminoácidos, con el fin de mejorar la solubilidad de los compuestos y lograr un mejor teñido del citoplasma.

\textcite{corona-lopezFarRedInfrared2021} propone la preparación de \gls{BOSCHIBA} a partir de la condensación de \reactant{2h1n} con la amina correspondiente; en esta investigación obtuvieron buenos rendimientos  (\qtyrange{85}{90}{\percent}), sin embargo el tiempo de reacción fue de \qty{48}{\hour}, por lo que llevar a cabo esta síntesis por medio de \gls{MW} en lugar de por reflujo convencional podría resultar en un menor tiempo de reacción.

El estudio realizado por \textcite{garcia-lopezNewLuminescentOrganoboron2022} detalla la síntesis de \gls{BOSCHIBA} tetracoordinados vía una reacción de condensación de tres componentes con rendimientos elevados \qtyrange{80}{90}{\percent} en un tiempo de reacción de \qty{20}{\minute}.

En una investigación donde se preparan bases de Schiff basadas en \ch{Sn} realizada por \textcite{lopez-espejelOrganotinSchiffBases2021} se conduce la reacción por medios convencionales de calentamiento y por \gls{MW}; el tiempo de reacción se vio drásticamente reducido, de \qty{24}{\hour} a \qty{3}{\minute}, y los rendimientos mejoraron. En este estudio también se evaluó el uso de bases de Schiff basadas en \ch{Sn} como agentes de tinción celular y se encontró que tienen baja citotoxicidad y buena capacidad de tinción.
