%arara: lualatex: { branch: developer, interaction: errorstopmode,
%arara: --> shell: yes, synctex: yes }
%arara: makeglossaries if found('aux', '@istfilename')
%arara: biber: { options: [ '--wraplines' ] }

% \DocumentMetadata{testphase=phase-III}
% \DocumentMetadata{lang=es-MX}

% NOTA: [<+->] se usa comio overlay para hacer que los bullets aparezcan uno por uno.

\documentclass[spanish,mexico]{scrartcl}
% \listfiles

\KOMAoptions{abstract=true}
\usepackage[spanish,mexico]{babel}


\usepackage{fontspec}
    \setmainfont{TeX Gyre Pagella}
    \setsansfont{TeX Gyre Heros}

\usepackage{csquotes}
% \MakeAutoQuote{"}{"}

\usepackage{microtype}
\setlength{\parskip}{1em}
\usepackage{tabularray}
\UseTblrLibrary{booktabs}
\UseTblrLibrary{siunitx}

\usepackage{siunitx}
\sisetup{separate-uncertainty,per-mode=symbol,detect-all,range-phrase=--}
\DeclareSIUnit{\angstrom}{\textup{\AA}}
\usepackage{chemmacros}
\usepackage[eps,librsvg]{chemobabel}
\usepackage{glossaries}
    \makeglossaries{}

\usepackage[style=science,backend=biber,url=false]{biblatex}
    \addbibresource[location=remote]{http://127.0.0.1:23119/better-bibtex/export/library?/1/library.biblatex}

\usepackage[skins]{tcolorbox}
\usepackage{paralist}
\usepackage{cleveref}
\usepackage{subcaption}
\usepackage[colorinlistoftodos]{todonotes}
\usepackage{lineno}
\usepackage{rotating}
\usepackage{newfloat}
\DeclareFloatingEnvironment[
   fileext=los,
   listname={List of Schemes},
   name=Scheme,
   placement=tbp,
   within=none % don't reset numbering
]{scheme}
\DeclareCaptionSubType{scheme}
\setcounter{secnumdepth}{5}

\begin{filecontents}[force]{abreviaturas.tex}
    \newacronym{DFT}{DFT}{Teoría del funcional de la densidad \textit{(del inglés "Density Functional Theory")}}
    \newacronym{TDDFT}{TDDFT}{Teoría del funcional de la densidad tiempo-dependiente \textit{(del inglés "Time-Dependant Density Functional Theory")}}
    \newacronym{FMR}{FMR}{Rotores Moleculares Fluorescentes \textit{(Del inglés Flurescent Molecular Rotor)}}
    \newacronym{BOSCHIBA}{BOSCHIBA}{Bases de Schiff de Boro \textit{(del inglés "\textsc{Bo}ron \textsc{Schi}ff \textsc{Ba}ses")}}
    \newacronym{BODIPY}{BODIPY}{\textsc{bo}ron-\textsc{di}\textsc{py}rromethene}
    \newacronym{TICT}{TICT}{transferencia de carga intramolecular retorcida \textit{(del inglés "twsited intramolecular charge transfer")}}
    \newacronym{LE}{LE}{Local Excitado \textit{(del inglés "locally exited")}}
    \newacronym{PES}{PES}{Superficie de Energía Potencial \textit{(del inglés "Potential Energy Surface")}}
    \newacronym{NBO}{NBO}{Orbitales Naturales de Enlace \textit{(del inglés "Natural Bond Orbitals")}}
    \newacronym{ICT}{ICT}{Transferencia de Carga Intramolecular \textit{(del inglés "Intramolecular Charge Transfer")}}
    \newacronym{VEE}{VEE}{Energía de Emisión Vertical \textit{(del inglés "Vertical Emission Energy")}}
    \newacronym{FMO}{FMO}{Orbitales Moleculares de Frontera \textit{(del inglés "Frontier Molecular Orbitals")}}
    \newacronym{HOMO}{HOMO}{Orbital Molecular de mas alta energía \textit{(del inglés "Highest Occupied Molecular Orbital")}}
    \newacronym{LUMO}{LUMO}{Orbital Molecular no ocupado de más baja energía \textit{(del inglés "Lowest Unoccupied Molecular Orbital")}}
    \newacronym{NMR}{NMR}{Resonancia Magnética Nuclear \textit{(del inglés "Nuclear Magnetic Resonance")}}
\end{filecontents}

\begin{filecontents}[force]{comandos.tex}
    % \newcommand{\invitro}{\textit{in-vitro}}
\end{filecontents}

\begin{filecontents}[force]{gantt.tex}
    \def\pgfcalendarweekdayletter#1{% \ifcase#1M\or T\or W\or T\or F\or S\or S\fi%
}
\begin{ganttchart}[
hgrid,
  vgrid,
  x unit=18mm,
  time slot format=little-endian
]{7.1.2013}{13.1.2013}
\gantttitlecalendar*{7.1.2013}{13.1.2013}{
month, month=name, month=shortname, weekday,
    weekday=name, weekday=shortname, weekday=letter
  }
\end{ganttchart}
\end{filecontents}

\begin{filecontents}[force]{quimica.tex}
    \DeclareChemReactant{BO-H}{name={BO-H}}
    \DeclareChemReactant{BO-trp}{name={BO-Trp}}
    \DeclareChemReactant{BO-phe}{name={BO-Phe}}
    \DeclareChemReactant{BO-tyr}{name={Bo-Tyr}}
    \DeclareChemReactant{BO-gly}{name={BO-Gly}}
    \DeclareChemReactant{trp}{name={triptófano}, short={trp}}
    \DeclareChemReactant{phe}{name={fenilalanina}, short={phe}}
    \DeclareChemReactant{tyr}{name={tirosina}, short={tyr}}
    \DeclareChemReactant{gly}{name={glicina}, short={gly}}
    \NewChemLatin\invitro{in vitro}
    \NewChemLatin\insilico{in silico}
    \DeclareChemTranslation{scheme-name}{spanish}{Esquema}
    \DeclareChemTranslation{scheme-list}{spanish}{Lista de esquemas}
    \DeclareChemTranslation{scheme}{spanish}{esquema}
    \DeclareChemTranslation{schemes}{spanish}{esquemas}
    \DeclareChemTranslation{Scheme}{spanish}{Esquema}
    \DeclareChemTranslation{Schemes}{spanish}{Esquemas}
    \renewcommand{\schemename}{Esquema}
\end{filecontents}

%% LaTeX2e file `abreviaturas.tex'
%% generated by the `filecontents' environment
%% from source `PEAG-Protocolo_maestria' on 2023/10/12.
%%
    \newacronym{DFT}{DFT}{Teoría del funcional de la densidad \textit{(del inglés "Density Functional Theory")}}
    \newacronym{TDDFT}{TDDFT}{Teoría del funcional de la densidad tiempo-dependiente \textit{(del inglés "Time-Dependant Density Functional Theory")}}
    \newacronym{FMR}{FMR}{Rotores Moleculares Fluorescentes \textit{(Del inglés Flurescent Molecular Rotor)}}
    \newacronym{BOSCHIBA}{BOSCHIBA}{Bases de Schiff de Boro \textit{(del inglés "\textsc{Bo}ron \textsc{Schi}ff \textsc{Ba}ses")}}
    \newacronym{BODIPY}{BODIPY}{\textsc{bo}ron-\textsc{di}\textsc{py}rromethene}
    \newacronym{TICT}{TICT}{transferencia de carga intramolecular retorcida \textit{(del inglés "twsited intramolecular charge transfer")}}
    \newacronym{LE}{LE}{Local Excitado \textit{(del inglés "locally exited")}}
    \newacronym{MW}{MW}{Microondas \textit{(del inglés "Microwave")}}
    \newacronym{PES}{PES}{Superficie de Energía Potencial \textit{(del inglés "Potential Energy Surface")}}
    \newacronym{NBO}{NBO}{Orbitales Naturales de Enlace \textit{(del inglés "Natural Bond Orbitals")}}
    \newacronym{ICT}{ICT}{Transferencia de Carga Intramolecular \textit{(del inglés "Intramolecular Charge Transfer")}}
    \newacronym{VEE}{VEE}{Energía de Emisión Vertical \textit{(del inglés "Vertical Emission Energy")}}
    \newacronym{FMO}{FMO}{Orbitales Moleculares de Frontera \textit{(del inglés "Frontier Molecular Orbitals")}}
    \newacronym{HOMO}{HOMO}{Orbital Molecular de mas alta energía \textit{(del inglés "Highest Occupied Molecular Orbital")}}
    \newacronym{LUMO}{LUMO}{Orbital Molecular no ocupado de más baja energía \textit{(del inglés "Lowest Unoccupied Molecular Orbital")}}
    \newacronym{NMR}{NMR}{Resonancia Magnética Nuclear \textit{(del inglés "Nuclear Magnetic Resonance")}}
    \newacronym{CREST}{CREST}{\textit{Conformer-Rotamer Ensemble Sampling Tool}}

%% LaTeX2e file `comandos.tex'
%% generated by the `filecontents' environment
%% from source `PEAG-Protocolo_maestria' on 2023/09/24.
%%
    % \newcommand{\invitro}{\textit{in-vitro}}
    \newcommand\scan{\(\text{r}^{2}\text{SCAN-3c}\)}

%% LaTeX2e file `quimica.tex'
%% generated by the `filecontents' environment
%% from source `PEAG-Protocolo_maestria' on 2023/11/28.
%%
    \usepackage{chemmacros}
    \DeclareChemReactant{2h1n}{name={\iupac{2-Hidroxi-1-naftaldehido}}, short={\iupac{2-Hidroxi-1-naftaldehido}}}
    \DeclareChemReactant{aphb}{name={ácido fenil borónico}, short={\ch{PhB(OH)2}}}
    \DeclareChemReactant{gly}{name={glicina}, short={gly}}
    \DeclareChemReactant{trp}{name={\iupac{\laevus-triptófano}}, short={trp}}
    \DeclareChemReactant{tyr}{name={\iupac{\laevus-tirosina}}, short={tyr}}
    \DeclareChemReactant{phe}{name={\iupac{\laevus-fenilalanina}}, short={phe}}
    \DeclareChemReactant{BO-gly}{name={BO-Gly}} % 1
    \DeclareChemReactant{BO-trp}{name={BO-Trp}} % 2
    \DeclareChemReactant{BO-tyr}{name={BO-Tyr}} % 3
    \DeclareChemReactant{BO-phe}{name={BO-Phe}} % 4
    \NewChemLatin\invitro{in vitro}
    \NewChemLatin\invivo{in vivo}
    \NewChemLatin\insilico{in silico}
    \DeclareChemTranslation{scheme-name}{spanish}{Esquema}
    \DeclareChemTranslation{scheme-list}{spanish}{Lista de esquemas}
    \DeclareChemTranslation{scheme}{spanish}{esquema}
    \DeclareChemTranslation{schemes}{spanish}{esquemas}
    \DeclareChemTranslation{Scheme}{spanish}{Esquema}
    \DeclareChemTranslation{Schemes}{spanish}{Esquemas}
    \chemsetup{language=spanish}
    \chemsetup[reactants]{printreactants-style=xltabular}
    % \renewcommand{\schemename}{Esquema}
    % \AtBeginDocument{
    % \renewcommand{\tablename}{Tabla}
    % }

    \AtBeginDocument{%
    \addto\captionsspanish{\renewcommand\listschemename{Índice de esquemas}}%
    \addto\captionsspanish{\renewcommand\schemename{Esquema}}%
    \csname captions\languagename\endcsname
}

    % \chemsetup[reactants]{printreactants-style=xltabular}


\title{Síntesis sustentable, caracterización química-fotofísica, y por {DFT} de {BOSCHIBA} derivadas de aminoácidos y su aplicación in vitro}
\subject{Protocolo de tesis de maestría}
\date{\today}
\author{Pablo E. Alanis González}
\publishers{Universidad Autónoma de Nuevo León, División de Posgrado}

\begin{document}
\linenumbers{}
\maketitle

\begin{abstract}
    Se sintetizarán una serie de \gls{BOSCHIBA} derivadas de \reactant*{trp}, \reactant*{phe}, \reactant*{tyr} y \reactant*{gly}. Se caracterizarán por métodos espectroscópicos. Se realizarán cálculos \insilico{} por medio de \gls{DFT} y \gls{TDDFT} para estudiar las propiedades fotofísicas de los compuestos y comprobar los mecanismos involucrados en el efecto supresor de la luminisencia en dichos compuestos así como estudios de topológicos sobre estos. A su vez, se realizarán estudios de citotoxicidad y tinción \invitro{} para determinar su actividad biológica de los compuestos.
\end{abstract}

\newpage
\tableofcontents
\listoffigures
\listoftables
\listofschemes

\section{Introducción}

Recientemente, se ha acrecentado el interés por los compuestos fluorescentes de boro debido a su amplio campo de aplicaciones; \autocite{ibarra-rodriguezOrganoboronSchiffBases2019} ya sea en sensores, como en tintas de seguridad, o bien los \gls{BODIPY} comercialmente disponibles utilizados como agentes para la tinción celular, ER-Tracker™ Green y ER-Tracker™ Red (ver \cref{ER-Trackers}).

\begin{scheme}
    \centering
    \begin{subscheme}{0.45\linewidth}
        \includegraphics[width=\linewidth]{ER-Tracker_Blue.pdf}
        \caption{ER-Tracker™ Blue}
        \label{ER-Tracker_Blue}
    \end{subscheme}
    \hfill
    \begin{subscheme}{0.45\linewidth}
        \includegraphics[width=\linewidth]{ER-Tracker_Green.pdf}
        \caption{ER-Tracker™ Green}
        \label{ER-Tracker_Green}
    \end{subscheme}
    \caption{Los ER-Tracker™ Green y ER-Tracker™ Red de Thermo Fischer Scientific™ son \gls{BODIPY} comerciales utilizados como agentes para la tinción celular.}
    \label{ER-Trackers}
\end{scheme}

Los \gls{FMR} son fluoróforos sensibles a la viscosidad que presentan una rotación libre que se vuelven fluorescentes, o en su defecto, aumentan la fluorescencia solo si su rotación se ve restringida.\todo{Citar} Algunas interacciones de carácter intramolecular para detener la rotación de los \gls{FMR} son \begin{inparaenum}[i.]
    \item formar interacciones de hidrógeno,\autocite{wuMultistageRotationalSpeed2018}
    \item a través del impedimento estérico,\autocite{faulknerAllostericRegulationRotational2016} o
    \item por la formación de complejos estables con iones metálicos.\autocite{yadavViscochromicMechanochromicUnsymmetrical2019}
\end{inparaenum}

Se ha determinado que la polarización del solvente y la viscosidad del mismo afectan considerablemente la fluorescencia de los \gls{FMR}. Esto es porque se limita la tasa de formación del complejo \gls{TICT}, el cual se de-exita de forma \emph{no-radiativa,} y se promueve la formación del complejo \gls{LE}, el cual fluórese. El efecto que tiene la polarización del solvente, aunque se sabe que es importante, no se ha logrado elucidar de forma aislada a la viscosidad.\cite{haidekkerEffectsSolventPolarity2005}

Diferentes estrategias para el diseño de \gls{FMR} se han propuesto para realizar sensores de viscosidad altamente sensibles, por ejemplo, incorporando grupos rotacionales asimétricos,\autocite{leePyrrolicMolecularRotors2016} rotadores con alta capacidad rotacional,\autocite{karpenkoPushPullDioxaborine2016} variación de puentes π-conjugados \emph{push-pull},\autocite{karpenkoPushPullDioxaborine2016} la aplicación de rotadores di- o trímeros,\autocite{kimballBODIPYBODIPYDyad2015} y la introducción de dos rotadores con diferentes capacidades rotacionales y electrondonantes.\autocite{rautTriazinebasedBODIPYTrimer2016}

Por lo que, obtener tanto una alta eficiencia de fluorescencia como un contraste fluorescente simultáneamente es muy difícil, debido a que, en muchos casos, las moléculas de alto rendimiento cuántico tienen una capacidad de contraste deficiente.
El rendimiento cuántico y el contraste de fluorescencia de los \gls{FMR} están inversamente correlacionados, una relación llamada "intensidad de fluorescencia---contraste".\autocite{leeFrontCoverFluorescent2018}

En la actualidad existe una amplia variedad de \gls{FMR} derivados de compuestos de boro, donde los \gls{BODIPY} y los dioxaborinos son los protagonistas debido a su elevado rendimiento cuántico, sin embargo, muestran algunas desventajas como la síntesis en varias etapas, condiciones de atmósfera anhidra y, en muchas ocasiones, una capacidad de contraste baja, aumentando escasamente el rendimiento cuántico del valor inicial.\autocite{karpenkoPushPullDioxaborine2016,guptaBodipyBasedFluorescent2016,liBODIPYBasedTwoPhotonFluorescent2018,kimBorondifluorideComplexesHemicurcuminoids2016}

Recientemente, nuestro grupo de trabajo ha informado sobre la síntesis de \gls{BOSCHIBA} y su uso como \gls{FMR} en la detección de viscosidad y la bioimagen de células.\autocite{ibarra-rodriguezFluorescentMolecularRotors2017} Los resultados encontrados indican que los \gls{BOSCHIBA} pueden aumentar hasta 34 veces su valor de rendimiento cuántico en medios de alta viscosidad, sin embargo, a pesar de teñir selectivamente el citoplasma en las células de melanoma, presentaron un bajo teñido atribuido principalmente a la baja solubilidad de los compuestos.

Para lograr mejorar el contraste de fluorescencia y la bioimagen celular, se diseñó una serie de \gls{BOSCHIBA} derivados de aminoácidos (ver \cref{sch:marcha}), donde las moléculas presentan rotación libre a través del anillo fenilborónico, y el aminoácido podría dar una mayor compatibilidad y solubilidad en medios celulares. Los compuestos de boro fluorescentes \textbf{1-4} se obtuvieron por irradiación ultrasónica en combinación con una reacción multicomponente con altos rendimientos químicos (\qty{>90}{\percent}) y un tiempo de reacción corto de \qty{20}{\minute} a \qty{50}{\degreeCelsius}. Este método resulta más eficiente y rápido en comparación con moléculas similares reportadas en la literatura.

\begin{scheme}
    \centering
    \includegraphics[width=0.5\linewidth]{Marcha.pdf}
    \caption{Compuestos que se sintetizarán en esta investigación.}
    \label{sch:marcha}
\end{scheme}
\todo{Esta figura no corresponde a los compuestos que se plantean en la introducción del proyecto.}

\section{Antecedentes}\todo{Hacer!} %Puede ser una tabla con resumen de lo que se ha hecho
\SetTblrInner{fontsize}{\fontsize{10}{12}\selectfont}
\begin{longtblr}[
        caption={Antecedentes de la investigación.},
        label={tbl:antecedentes}
    ]{
        colspec = {X[2,c,m] X[2,c,m] X[c,m]},
        rowhead = 1,
        cells   = {font = \fontsize{8pt}{10pt}\selectfont},
        row{1} = {font=\bfseries}
    }
    \toprule
    Investigación                                                       & Aportación                                                       & Referencia                                                    \\ \midrule
    \citetitle*{lopez-espejelOrganotinSchiffBases2021}                  & Sintesis de \gls{BOSCHIBA} con \ch{Sn}                           & \cite{lopez-espejelOrganotinSchiffBases2021}                  \\
    \citetitle*{corona-lopezFarRedInfrared2021}                         & Síntesis de \gls{BOSCHIBA} con clusters de boro                  & \cite{corona-lopezFarRedInfrared2021}                         \\
    \citetitle*{ibarra-rodriguezOrganoboronSchiffBases2019}             & Síntesis de \gls{BOSCHIBA} y su uso como sondas fluorescentes    & \cite{ibarra-rodriguezOrganoboronSchiffBases2019}             \\
    \citetitle*{canton-diazOnepotMicrowaveassistedSynthesis2018}        & Sintesis \textit{one-pot} de \gls{BOSCHIBA}                      & \cite{canton-diazOnepotMicrowaveassistedSynthesis2018}        \\
    \citetitle*{corona-lopezSynthesisCharacterizationPhotophysical2017} & Síntesis de \gls{BOSCHIBA} y su aplicación para teñir citoplasma & \cite{corona-lopezSynthesisCharacterizationPhotophysical2017} \\
    \bottomrule
\end{longtblr}

\subsection{Análisis crítico de los antecedentes}\todo{Hacer!} % entrar en detalle de lo que se ha hecho.

\section{Aportación científica}\todo{Verificar}
Plantear una metodología para la sintesis de compuestos de boro fluorescentes, con un alto rendimiento cuántico y un alto contraste de fluorescencia en medios de alta viscosidad, a partir de aminoácidos, así como su aplicación en la bioimagen de células. También se realizarán estudios \insilico{} para determinar las propiedades fotofísicas de los compuestos.

\section{Hipótesis}\todo{Verificar}
La incorporación de aminoácidos en la estructura de los \gls{BOSCHIBA} logrará una mejor penetración de las membranas celulares, además de que se espera que los compuestos presenten un alto rendimiento cuántico y un alto contraste de fluorescencia en medios de alta viscosidad.

\section{Objetivos y metas}\todo{Verificar}
\subsection{Objetivo general}\todo{Verificar!}
Realizar la síntesis de una serie de \gls{BOSCHIBA} con su posible aplicación en tinción celular y estudiar sus propiedades fotofísicas por medio de cálculos \insilico{}.

\subsection{Objetivos específicos}\todo{Verificar!}
\begin{itemize}
    \item \textbf{Sintetizar} una serie de \gls{BOSCHIBA} derivadas de \reactant{trp}, \reactant{phe}, \reactant{tyr} y \reactant{gly};
    \item \textbf{Elucidar} los mecanismos involucrados en el efecto supresor de la luminiscencia en \reactant{BO-trp};
    \item \textbf{Caracterizar} los compuestos por métodos espectroscópicos;
\end{itemize}

\section{Metodología}\todo{Hacer!}
\subsection{Materiales}
\begin{longtblr}[
        caption = {Equipos que se utilizarán para la caracterización de los compuestos de esta investigación.},
        label = {tbl:equipos}
    ]{
        colspec = {X[2,c,m] X[c,m] X[c,m]},
    }
    \toprule
    \textbf{Instrumento}                             & \textbf{ Marca y modelo} & \textbf{Locación}          \\ \midrule
    Espectrofotómetro                                & {Perkin-Elmer,                                        \\Lambda 365} & FCQ \\
    Espectrofluorímetro                              & {Horriba Scientificm,                                 \\Fluorolog-3} & FCQ \\
    Sistema LC/MS/MS                                 & {EMAR,                                                \\AB Sciex API 2000} & CINVESTAV---IPN (DF) \\
    Difractómetro de rayos X                         & {Bruker,                                              \\SMART APEX CCD} & CINVESTAV---IPN (DF) \\
    \gls{NMR} (\NMR*{1,H}, \NMR*{11,B}, \NMR*{13,C}) & {Bruker,                                              \\Avance DPX 400} & Facultad de Medicina, UANL \\
    Microscopio confocal                             & ---                      & Facultad de Biología, UANL \\
    \bottomrule
\end{longtblr}

\subsection{Experimental}
\subsubsection{Síntesis}\todo{Hacer!}
\subsubsection{Determinación de propiedades ópticas}\todo{Hacer!}
\subsubsection{Determinación de citotoxicidad}\todo{Hacer!}
\subsubsection{Modelado molecular}\todo{Hacer!}
\paragraph{Selección del funcional}\todo{Hacer!}
\paragraph{Optimización geométrica}\todo{Hacer!}
\paragraph{Obtención de espectro de emisión}\todo{Hacer!}
\subsection{Disposición de resiudos}
\begin{longtblr}[
        caption = {Residuos que se generarán derivados de esta investigación.},
        label = {tbl:residuos}
    ]{
        colspec = {X[2,c,m] X[2,c,m] X[c,m]},
    }
    \toprule
    \textbf{Resiudo}  & \textbf{Tipo}                   & \textbf{Disposición} \\ \midrule
    \ch{MeOH}, hexano & Solventes orgánicos             & C                    \\
    DCM, Cloroformo   & Solventes orgánicos halogenados & C                    \\
    Sales inorgánicas & Sales                           & A                    \\
    \bottomrule
\end{longtblr}

\subsection{Costos}
Se estima un costo aproximado de \num{70000} MXN para la realización de esta investigación, el cual se desglosa en la \cref{tbl:costos}.\todo{preguntar por costos de las tecnicas de caracterizacion y miscelanos}

% \section{Residuos}\todo{Hacer!}
% \section{Análisis de costos}\todo{Hacer!}
% \section{Procedimiento experimental}\todo{Hacer!}
\begin{sidewaysfigure}
    \section{Planificación}
    \includegraphics[width=0.9\linewidth]{gantt.pdf}
    \caption{Diagrama de Gantt de la investigación.}
    \label{fig:gantt}
\end{sidewaysfigure}
\printreactants{}
\printglossaries{}
\printbibliography{}

\listoftodos[Pendientes]

\end{document}
